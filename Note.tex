\documentclass{article}

\usepackage{cross-and-crescent}
\usepackage[no-config]{fontspec}
\usepackage{url}

\input support/note-info
\input support/donor-info

% This note is going in a double-windowed No. 9 envelope:
% https://www.uline.com/Product/Detail/S-17269/Paper-Office-Envelopes/9-Self-Seal-White-Business-Envelopes-with-Double-Window-3-7-8-x-8-7-8
% The envelope is 3.875 inches high and 8.875 inches wide. The top window is
%   - 0.75  inches from the top of the envelope (2 inches from the bottom)
%   - 3.5   inches wide
%   - 1.125 inches high
% The bottom window is
%   - 2.375 inches from the top of the envelope (0.5 inches from the bottom)
%   - 4     inches wide
%   - 1     inch   high
% Both windows are 0.375 inches from the left of the envelope.
\newlength\topWindowMinY
\topWindowMinY0.75in
\newlength\topWindowWidth
\topWindowWidth3.5in
\newlength\topWindowHeight
\topWindowHeight1.125in

\newlength\bottomWindowMinY
\bottomWindowMinY2.375in
\newlength\bottomWindowWidth
\bottomWindowWidth4in
\newlength\bottomWindowHeight
\bottomWindowHeight1in

% Set up margins.
\usepackage[
  left   = 1.25in,
  right  = 2.375in,
  top    = \topWindowMinY,
  bottom = 1.25in,
  noheadfoot,
  nomarginpar
]{geometry}

% Use Linux Libertine <http://www.linuxlibertine.org> as the main font.
\setmainfont{Linux Libertine O}[Numbers=Proportional]

% Use Gillius <http://arkandis.tuxfamily.org/adffonts.html> as the sans serif
% font.
\setsansfont{Gillius ADF No2}

% Don’t indent paragraphs.
\parindent0sp

% Separate paragraphs by 6 pt (“bp” means “big point”, and 1bp is 1/72 inch).
\parskip6bp

\begin{document}
\frenchspacing    % Don’t put extra space after a full stop.
\pagestyle{empty} % Don’t put a page number in the footer.
\urlstyle{same}   % Don’t typeset URLs in a fixed-width font.

% Add negative horizontal space so that a cross and crescent can be placed about
% a half inch into the left margin.
\newlength\crossAndCrescentSize
\crossAndCrescentSize0.5in
\newlength\crossAndCrescentThickness
\crossAndCrescentThickness\pgflinewidth
\newlength\crossAndCrescentSeparation
\crossAndCrescentSeparation1ex
\hspace*{\dimexpr-\crossAndCrescentSize-\crossAndCrescentThickness-\crossAndCrescentSeparation\relax}%
% Put a cross and crescent and the return address in a box.
\raisebox{0sp}[%
    % Set the height of the box so that its contents are centered in the top
    % window.
    \dimexpr(\topWindowHeight + \height) / 2\relax%
  ][%
    % Set the depth of the box so that the bottom of the box is the bottom of
    % the window.
    \dimexpr(\topWindowHeight - \height) / 2\relax%
  ]{%
  % Draw a cross and crescent in the left margin. To draw the cross and crescent
  % at the desired size, it must be scaled. The path of the cross and crescent
  % is 8 by 8 units in TikZ’s coordinate system. By default, a unit in TikZ is
  % 1 centimeter (see Section 11.1 of the TikZ & PGF manual for v3.0.1a at
  % https://www.ctan.org/pkg/pgf).
  \begin{tikzpicture}[scale=\crossAndCrescentSize / 8cm]%
    \crossAndCrescentSetMacros
    \draw[line width=\crossAndCrescentThickness] \crossAndCrescentPath
  \end{tikzpicture}%
  \hspace{\crossAndCrescentSeparation}%
  % Add the return address.
  \parbox[b][\crossAndCrescentSize][c]{\topWindowWidth}{%
    Sigma Alumni Association\\
    \SigmaStreet\\
    \SigmaCityStateAndZIP%
  }%
}%
% Add space to put the mailing address in the bottom window.
\\[\dimexpr\bottomWindowMinY-\topWindowHeight-\topWindowMinY\relax]%
% Add a box for the mailing address.
\parbox[t][\bottomWindowHeight][c]{\bottomWindowWidth}{%
  \sffamily\fontsize{12}{15}\selectfont%
  \MakeUppercase{\donorDisplayName}\\%
  \MakeUppercase{\donorStreet}
\\%
  % From https://pe.usps.com/text/pub28/28c2_009.htm, put two spaces between the
  % state and ZIP.
  \MakeUppercase{\donorCity\ \donorState\ \ \donorZIP}%
}%

\vfill

\today

\vspace{0.25in}

Dear Brother \donorLastName,

Thank you for your non–tax-deductible gift of \donationAmount\ to Sigma Alumni
Association. We will use your gift to provide the best possible Fraternal
experience to undergraduate brothers of Sigma Zeta, the University of Michigan
chapter of ΛΧΑ. Your gift \emph{will} make a difference. Please write to
info@lcamichigan.com if you have any questions about your gift. Again, thank
you!

\vfill

{\fontsize{14}{22}\selectfont\addfontfeature{Numbers=OldStyle}
  We hope to see you at\\[8bp]
  {\sffamily\fontsize{24}{24}\selectfont\addfontfeature{LetterSpace=5}
    \MakeUppercase{\eventName}%
  }\\
  on \eventDate\ at 1601 Washtenaw.\\
  RSVP to info@lcamichigan.com.\par
}

\vspace{0.25in}

\emph{Send alumni news and contributions to\\}
\SigmaStreet, \SigmaCityStateAndZIP

\emph{Subscribe to the email list at} \url{lcamichigan.com/alumni-news}

\emph{Send an update for a future newsletter at\\}
\url{lcamichigan.com/alumni-news#sigman}

\emph{Like Sigma at} \url{facebook.com/lcasigmazeta}

\emph{Join the LinkedIn group at} \url{linkedin.com/groups?gid=5072620}

% Change \iffalse to \iftrue, and then run
%   xetex -fmt=xelatex Note.tex
% *twice* to show the top and bottom envelope windows as gray rectangles.
\iffalse
\newlength\windowMinX
\windowMinX0.375in
\begin{tikzpicture}[remember picture,overlay]
  \begin{scope}[lightgray]
    \draw
      ([shift={(\windowMinX, -\topWindowMinY)}]current page.north west) rectangle
      ([shift={(\windowMinX + \topWindowWidth, -(\topWindowMinY + \topWindowHeight))}]current page.north west);
    \draw
      ([shift={(\windowMinX, -\bottomWindowMinY)}]current page.north west) rectangle
      ([shift={(\windowMinX + \bottomWindowWidth, -(\bottomWindowMinY + \bottomWindowHeight))}]current page.north west);
  \end{scope}
\end{tikzpicture}
\fi

\end{document}
